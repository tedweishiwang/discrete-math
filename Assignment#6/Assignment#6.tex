\documentclass{article} 

\usepackage{amsmath,amsthm,amssymb}

\newtheorem{problem}{Problem} 

\theoremstyle{definition} 

\newtheorem*{solution}{Solution} 

\begin{document} \title{Assignment 6} 

\author{Weishi Wang, ID 260540022} 

\date{\today}

\maketitle

\begin{problem} 
(a) Let a be any integer. Prove that \(a^n - an + n - 1\) is divisible by \((a-1)^2\) when \(n \ge 2\).\\\\

(b) We saw in class that\\

\(\sum_{k=1}^{n}{k} = \frac{n(n+1)}{2}\)\\

if we look at the following two sums, one might see a pattern emerging:\\

\(\sum_{k=1}^{n}{k(k+1)} = \frac{n(n+1)(n+2)}{3}\)\\

\(\sum_{k=1}^{n}{k(k+1)(k+2)} = \frac{n(n+1)(n+2)(n+3)}{4}\)\\

Prove the following, which generalizes the three summations above, for \(n \ge 1\) where \(m \ge 0\) is some fixed integer:\\

\(\sum_{k=1}^{n}{\frac{(k+m)!}{(k-1)!}} = \frac{(n+m+1)!}{(n-1)!(m+2)}\)\\\\

(c) You know about binary representations of integers, and I've asked you to prove things
about integers in base 10. Now, show that every positive integer has a factorial representation.
That is, prove that for every integer \(n \ge 1\), we can write\\

\(n = \sum_{i=1}^{k}{c_ii!}\)\\

where the integer coefficients \(c_i\) satisfy \(0 \le c_i \le i\) for each i.\\\\\\

\end{problem}




\begin{solution}
(a) Proof by induction.

The base case is n = 2:

\(a^2 - 2a + 2 - 1 = a^2 - 2a + 1 = (a-1)^2\)

\((a-1)^2\) is divisible by (a-1).

The base case is satisfied.\\

Inductive step:

Assume that \(a^n - an + n - 1\) is divisible by (a-1).

Then, check n+1:

\(a^{n+1} - a(n+1) + n + 1 - 1\)

=\(a^{n+1} + a^2n - a^2n + (n-1)a - (n-1)a - a(n+1) + n\)

=\(a^{n+1} - a^2n + (n-1)a + a^2n - (n-1)a - a(n+1) + n\)

=\(a[a^{n} - 2n + (n-1)] + a^2n - (n-1)a - a(n+1) + n\)

=\(a[a^{n} - 2n + (n-1)] + a^2n -2na + n\)

=\(a[a^{n} - 2n + (n-1)] + n(a-1)^2\)

From the assumption, we know that \(a[a^{n} - 2n + (n-1)]\) is divisible by (a-1), also we know that \(n(a-1)^2\) is divisible by (a-1).

Therefore, \(a[a^{n} - 2n + (n-1)] + n(a-1)^2\) is divisible by (a-1). 

Thus the claim is true for n+1 case, and therefore true for all \(n \ge 2\).\\\\





(b) Proof by induction.

The base case is n=1.

\(\sum_{k=1}^{1}{\frac{(k+m)!}{(k-1)!}} = \frac{(1+m)!}{0!} = (1+m)!\)

Also, when n=1, \(\frac{(n+m+1)!}{(n-1)!(m+2)} = \frac{(m+2)!}{(0)!(m+2)} = (m+1)!\)

They are equal, which proves the base case (n=1).\\

Inductive step:

Assume that \(\sum_{k=1}^{n}{\frac{(k+m)!}{(k-1)!}} = \frac{(n+m+1)!}{(n-1)!(m+2)}\) is true.

Now, check n+1 case:

\(\sum_{k=1}^{n+1}{\frac{(k+m)!}{(k-1)!}}\) 

\(= \sum_{k=1}^{n}{\frac{(k+m)!}{(k-1)!}} + \frac{(n+m+1)!}{n!}\)

by the inductive hypothesis of case n, we obtain:

\(= \frac{(n+m+1)!}{(n-1)!(m+2)} + \frac{(n+m+1)!}{n!}\)

\(= \frac{(n+m+1)!(n+m+2)}{n!(m+2)}\)

\(= \frac{(n+m+2)!}{n!(m+2)}\)

Which means that it holds for n+1 case.

Thus, the claim is true by induction.\\\\\\


(c) First, write down some of first few therms to find a pattern:

\(0 = 0(0!)\)

\(1 = 0(0!) + 1(1!)\)

\(2 = 0(0!) + 0(1!) + 1(2!)\)

\(3 = 0(0!) + 1(1!) + 1(2!)\)

\(4 = 0(0!) + 0(1!) + 2(2!)\)

\(5 = 0(0!) + 1(1!) + 2(2!)\)

\(6 = 0(0!) + 0(1!) + 0(2!) + 1(3!)\)

\(7 = 0(0!) + 1(1!) + 0(2!) + 1(3!)\)

something that we notice is that if we can increment by 1 each time, we can represent every integer.

Thus, we only need to prove that:

 \(0(0!) + 0(1!) + 0(2!) + ... + 0(n!) + (1)(n+1)! = 0(0!) + 1(1!) + 2(2!) + 3(3!) + ... + n(n!)\)

Which implies a increment of 1.\\

We can prove this by induction.

Base case: n = 0,

\(0(0!) = 0(0!) + 1(1!) - 1\)

The base case satisfies the claim.\\

Inductive step:

Assume the claim is true, then we need to prove that it is true for n+1 case as well.

\(0(0!) + 1(1!) + 2(2!) + 3(3!) + ... + n(n!) + (n+1)(n+1)!\)

\(= (n+1)! -1 + (n+1)(n+1)!\)

\(= (n+2)(n+1)! - 1\)

\(= (n+2)! - 1\)

Which proves that n+1 case is true.

Thus, by induction, the claim is true, and therefore, there is a factorial representation for all integers greater or equals to 0.\\\\\\


\end{solution}

\begin{problem}
(a) Prove that \(f_1 - f_2 + f_3 + ... + (-1)^nf_{n+1} = (-1)^nf_n + 1\) for all \(n \ge 1\).

(b) Prove that \(f_1f_2 + f_2f_3 + f_3f_4 + ... + f_{2n-1}f_{2n} = f_{2n}^2\) for all \(n \ge 1\).\\\\\\

\end{problem}

\begin{solution}
(a) Proof by induction.

The base case is n = 1:

\(f_1-f_2 = 1 - 1 = 0\)

\( -f_1 + 1 = -1 + 1 = 0\)

So, \(f_1-f_2 =  -f_1 + 1\)

which means that the claim holds for the base case.\\

Inductive step:

Assume that \(f_1 - f_2 + f_3 + ... + (-1)^nf_{n+1} = (-1)^nf_n + 1\) is true for n.

Check n+1 case:

We can separate this into two cases:\\

1. If n is odd,

\(f_1 - f_2 + f_3 + ...  - f_{n+1} + f_{n+2} \)

\(=  -f_n + 1 + f_{n+2}\)

\(=  f_{n+1} + 1\)

which is effectively \((-1)^{n+1}f_{n+1} + 1\) when n is odd.\\


2. If n is even,

\(f_1 - f_2 + f_3 + ...  + f_{n+1} - f_{n+2} \)

\(=  f_n + 1 - f_{n+2}\)

\(=  -f_{n+1} + 1\)

which is again \((-1)^{n+1}f_{n+1} + 1\) when n is even.\\

Combining 2 cases, we conclude that the original claim is true for n+1 case.

Thus, the claim is true for all \(n \ge 1\).\\\\\\


(b) Proof by induction again.

The base case is n=1,

\(f_1f_2 = 1 = f_2^2\)

This proves that the claim holds for the base case.\\

Inductive step:

Assume that \(f_1f_2 + f_2f_3 + f_3f_4 + ... + f_{2n-1}f_{2n} = f_{2n}^2\) is true for n.

Check n+1 case:

\(f_1f_2 + f_2f_3 + f_3f_4 + ... + f_{2n-1}f_{2n}  +  f_{2n}f_{2n+1} +  f_{2n+1}f_{2n+2}\)

\(= f_{2n}^2 +  f_{2n}f_{2n+1} +  f_{2n+1}f_{2n+2}\) 

\(= f_{2n}(f_{2n} + f_{2n+1}) +  f_{2n+1}f_{2n+2}\) 

\(= f_{2n}f_{2n+2} +  f_{2n+1}f_{2n+2}\) 

\(= (f_{2n} +  f_{2n+1})f_{2n+2}\) 

\(= f_{2n+2}f_{2n+2}\) 

\(= f_{2n+2}^2\) \\

Therefore, the claim holds for n+1 case.

Thus, the claim is true for all \(n \ge 1\).\\\\\\

\end{solution}


\begin{problem}
Recurrence relations. You've won a contest! You're going to win money! Your prize is
determined as follows. You are given \(\$40\), then asked to sit in a chair. At each minute mark
of you being in the chair, your winnings are re-calculated as being 150 \(\%\) of the amount you
held during the previous minute but deducted from that is 25 \( \%\) of the amount you held the
minute before that (note that you held \(\$0\) before the contest started). Whoever is holding
the contest is no fool; it's not hard to see that there needs to be some cost to you sitting in
the chair, or they'll go bankrupt! So, at each minute mark, you're going to lose \(\$\)6 for every
minute you've been in the chair (after the first minute you'll lose \(\$6\), after the second minute
you'll lose an another \(\$\)12, after the third minute you'll lose another \(\$\)18, and so on). You
can leave the chair any time you want, collect your winnings, and walk away.\\\\

(c) Write a new recurrence relation that expresses the amount of money you win if you leave
the chair after the \(m^{th}\) minute.\\

(d) Solve this recurrence relation to find an explicit function of m for your winnings after m minutes.\\

(e) How long should you stay in the chair to maximize your winnings? If you make any
claims about the behaviour of the function after a given point, make sure you justify
your answer (this can be done using basic calculus or by other means).\\\\

\end{problem}






\begin{solution}

(c) Let \(a_m\) represent the money hold on \(m^{th}\) minutes.

Then, according to the rule, the following recurrence can be written:\\

\(a_m = (1+50\%)a_{m-1} - (25\%)a_{m-2} - 6m\)

  \(= 1.5a_{m-1} - 0.25a_{m-2} - 6m\)\\
  
  and \(a_0 = 40, a_1 = 1.5(40)-0-24 = 54\)\\\\\\




(d) First, find the homogeneous solution to this recurrence.

Multiply the equation both sides by 4.

\(4a_m = 6a_{m-1} - a_{m-2} - 24m\)

Ignore the 24m to find the homogeneous solution:

\(4a_m = 6a_{m-1} - a_{m-2}\)

\(4x^2 - 6x + 1 = 0\)

\(x = \frac{3 + \sqrt{5}}{4}\) or  \(x =  \frac{3 - \sqrt{5}}{4}\)

so the homogeneous solution is:

\(a_m = C_1(\frac{3 + \sqrt{5}}{4})^m + C_2(\frac{3 - \sqrt{5}}{4})^m\)\\

Now, find the particular solution:

Guess \(p_m = am + b\)

\(4(am+b) = 6[a(m -1)] -[a(m-2) + b] - 24m\)

\((a-24)m + (b-4a) = 0\)

The coefficients of m and constant must be 0, so:

a = 24 and b =96

So, \(p_m = 24m + 96\)

The solution is homogeneous + particular:

\(a_m = C_1(\frac{3 + \sqrt{5}}{4})^m + C_2(\frac{3 - \sqrt{5}}{4})^m + (24m + 96)\)\\

We can plug the initial conditions to find the coefficients \(C_1\) and \(C_2\).

\(a_0 = C_1 + C_2 + 96 = 40\)

\(a_1 = C_1(\frac{3 + \sqrt{5}}{4}) + C_2(x =  \frac{3 - \sqrt{5}}{4}) + (120) = 54\)\\

Solve for \(C_1\) and \(C_2\):

\(C_2 = \frac{48\sqrt{5}}{5} - 28 \) and \(C_1 = \frac{-48\sqrt{5}}{5} - 28 \)\\

Therefore, the solution to this recurrence is:

\(a_m = (\frac{-48\sqrt{5}}{5} - 28)(\frac{3 + \sqrt{5}}{4})^m + (\frac{48\sqrt{5}}{5} - 28)(\frac{3 - \sqrt{5}}{4})^m + (24m + 96)\)\\\\\\


(e) Find the derivative of \(a_m\) and make it equals to 0.

So, \(\frac{da_m}{dm} = 0\)

Solve the equation and the answer is approximately 2. Thus the value of m to maximize the price is m = 2.

To verify this, we can calculate the answer for m = 2.

\(a_2 = 1.5(54) - 0.25(40) - 12 = 59\)

Now calculate m = 3.

\(a_2 = 1.5(59) - 0.25(54) - 18 = 57\)

Which is smaller than \(a_2\)

Therefore, the value is maximized when m = 2.








\end{solution}





\end{document}