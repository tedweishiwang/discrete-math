\documentclass{article} 

\usepackage{amsmath,amsthm,amssymb}

\newtheorem{problem}{Problem} 

\theoremstyle{definition} 

\newtheorem*{solution}{Solution} 

\begin{document} \title{Assignment 1} 

\author{Weishi Wang, ID 260540022} 

\date{\today}

\maketitle

\begin{problem} 

Use a truth table to determine if each statement is a tautology, contradiction, or contingency.

(a) (P \(\vee\) Q) \(\Rightarrow\) \(\neg\)P \\

(b) (P \(\Leftrightarrow\) Q) \(\wedge\) (Q \(\Leftrightarrow\) R) \(\Rightarrow\) (P \(\Leftrightarrow\) R)\\

(c) [(P \(\oplus\) Q) \(\oplus\) \(\neg\)Q] \(\Leftrightarrow\) P, where P \(\oplus\) Q is defined by the following truth table:\\

\begin{displaymath}
\begin{array}{ c|c|c }
% |c c|c| means that there are three columns in the table and
% a vertical bar '|' will be printed on the left and right borders,
% and between the second and the third columns.
% The letter 'c' means the value will be centered within the column,
% letter 'l', left-aligned, and 'r', right-aligned.
P & Q & P \oplus Q \\ % Use & to separate the columns
\hline % Put a horizontal line between the table header and the rest.
T & T & F\\
T & F & T\\
F & T & T\\
F & F & F\\
\end{array}
\end{displaymath}
\\
\end{problem}


\begin{solution} Solution for problem 1:\\

(a)
\begin{displaymath}
\begin{array}{ c|c|c|c|c }
% |c c|c| means that there are three columns in the table and
% a vertical bar '|' will be printed on the left and right borders,
% and between the second and the third columns.
% The letter 'c' means the value will be centered within the column,
% letter 'l', left-aligned, and 'r', right-aligned.
P & Q & \neg P & P \vee Q & (P \vee Q) \Rightarrow \neg P  \\ % Use & to separate the columns
\hline % Put a horizontal line between the table header and the rest.
T & T & F & T & F\\
T & F & F & T & F\\
F & T & T & T & T\\
F & F & T & F & T\\
\end{array}
\end{displaymath}

Thus, this is a contingency.\\\\


(b)
\begin{displaymath}
\begin{array}{ c|c|c|c|c|c|c|c }
% |c c|c| means that there are three columns in the table and
% a vertical bar '|' will be printed on the left and right borders,
% and between the second and the third columns.
% The letter 'c' means the value will be centered within the column,
% letter 'l', left-aligned, and 'r', right-aligned.
P & Q & R & P \Leftrightarrow Q & Q \Leftrightarrow R & P \Leftrightarrow R &  (P \Leftrightarrow Q) \wedge (Q \Leftrightarrow R) &  (P \Leftrightarrow Q) \wedge (Q \Leftrightarrow R) \Rightarrow (P \Leftrightarrow R)  \\ % Use & to separate the columns
\hline % Put a horizontal line between the table header and the rest.
T & T & T & T & T & T & T & T\\
T & T & F & T & F & F & F & T\\
T & F & T & F & F & T & T & T\\
T & F & F & F & T & F & F & T\\
F & T & T & F & T & F & F & T\\
F & T & F & F & F & T & T & T\\
F & F & T & T & F & F & F & T\\
F & F & F & T & T & T & T & T\\
\end{array}
\end{displaymath}
Thus, This is a tautology.\\\\


(c)
\begin{displaymath}
\begin{array}{ c|c|c|c|c|c }
% |c c|c| means that there are three columns in the table and
% a vertical bar '|' will be printed on the left and right borders,
% and between the second and the third columns.
% The letter 'c' means the value will be centered within the column,
% letter 'l', left-aligned, and 'r', right-aligned.
P & Q & P \oplus Q & \neg Q & (P \oplus Q) \oplus \neg Q  & [(P \oplus Q) \oplus \neg Q] \Leftrightarrow P\\ % Use & to separate the columns
\hline % Put a horizontal line between the table header and the rest.
T & T & F & F & F & F\\
T & F & T & T & F & F\\
F & T & T & F & T & F\\
F & F & F & T & T & F\\
\end{array}
\end{displaymath}
Thus, this is a contradiction.

\break

\end{solution}


\begin{problem}

Verify the following statements using only identities (see the list posted on MyCourese). Show all of your work and name the identity or identites used in each step.

(a) [(P \(\Rightarrow\) Q) \(\wedge\) P] \(\Rightarrow\) Q is a tautology\\

(b) \(\neg\)(P \(\wedge\) Q) \(\wedge\) (Q \(\Rightarrow\) P) \(\equiv\) \(\neg\)Q\\

(c) \(\neg\)[(P \(\vee\) Q) \(\vee\) [(Q \(\vee\) \(\neg\)R) \(\wedge\) (P \(\vee\) R)]] \(\equiv\) \(\neg\)P \(\wedge\) \(\neg\)Q
\\
\\
\end{problem}


\begin{solution}
Solution for problem 2:\\

(a)\\

[(P \(\Rightarrow\) Q) \(\wedge\) P] \(\Rightarrow\) Q

\(\equiv\) [(\(\neg\)P \(\vee\) Q) \(\wedge\) P] \(\Rightarrow\) Q (Conditional)

\(\equiv\) [(\(\neg\)P \(\wedge\) P) \(\vee\) (Q \(\wedge\) P)] \(\Rightarrow\) Q (Distributive)

\(\equiv\) [(\(\mathbb{F}\)) \(\vee\) (Q \(\wedge\) P)] \(\Rightarrow\) Q (Complement)

\(\equiv\) (Q \(\wedge\) P) \(\Rightarrow\) Q (Identity)

\(\equiv\) \(\neg\)(Q \(\wedge\) P) \(\vee\) Q (Conditional)

\(\equiv\) (\(\neg\)Q \(\vee\) \(\neg\)P) \(\vee\) Q (DeMorgan's laws)

\(\equiv\) \(\neg\)Q \(\vee\) Q \(\vee\) \(\neg\)P (Associative)

\(\equiv\) \(\mathbb{T}\) \(\vee\) \(\neg\)P (Complement)

\(\equiv\) \(\mathbb{T}\) (Domination)

Thus, this is indeed a tautology.\\\\

(b)\\

\(\neg\)(P \(\wedge\) Q) \(\wedge\) (Q \(\Rightarrow\) P)

\(\equiv\) (\(\neg\)P \(\vee\) \(\neg\)Q) \(\wedge\) (Q \(\Rightarrow\) P) (DeMorgan's laws)

\(\equiv\) (\(\neg\)P \(\vee\) \(\neg\)Q) \(\wedge\) (\(\neg\)Q \(\vee\) P) (Conditional)

\(\equiv\) \(\neg\)Q \(\vee\) (\(\neg\)P \(\wedge\) P) (Factorization)

\(\equiv\) \(\neg\)Q \(\vee\) \(\mathbb{F}\) (Complement)

\(\equiv\) \(\neg\)Q (Identity)\\\\

(c)\\

\(\neg\)[(P \(\vee\) Q) \(\vee\) [(Q \(\vee\) \(\neg\)R) \(\wedge\) (P \(\vee\) R)]]

\(\equiv\) \(\neg\)[[(P \(\vee\) Q) \(\vee\) (Q \(\vee\) \(\neg\)R)] \(\wedge\) [(P \(\vee\) Q) \(\vee\) (P \(\vee\) R)]] (Distributive)

\(\equiv\) \(\neg\)[(P \(\vee\) Q \(\vee\) Q \(\vee\) \(\neg\)R) \(\wedge\) (P \(\vee\) Q \(\vee\) P \(\vee\) R)] (Associative)

\(\equiv\) \(\neg\)[(P \(\vee\) Q \(\vee\) \(\neg\)R) \(\wedge\) (P \(\vee\) Q \(\vee\) R)] (Idempotent)

\(\equiv\) \(\neg\)[(P \(\vee\) Q) \(\vee\) (\(\neg\)R \(\wedge\) R)] (Factorization)

\(\equiv\) \(\neg\)[(P \(\wedge\) Q) \(\vee\) \(\mathbb{F}\)] (Complement)

\(\equiv\) \(\neg\)(P \(\vee\) Q) (Identity)

\(\equiv\) \(\neg\)P \(\wedge\) \(\neg\)Q (DeMorgan's laws)

\end{solution}

\break

\begin{problem}
Of the following conditional and biconditional statements, which are true and which are false? Briefly justify your answers.

(a) \(\pi\) is an integer if and only if \(\sqrt{e+3}\) is a vowel.\\

(b) 0 $>$ 1 whenever 2 + 2 = 4.\\

(c) If (a) implies (b), then pigs cannot fly.\\\\
\end{problem}


\begin{solution} Solution for problem 3:\\

(a)\\

Let P = ``\(\sqrt{e+3}\) is a vowel" and Q = ``\(\pi\) is an integer".

The statement claims that P \(\Leftrightarrow\) Q.

 P \(\Leftrightarrow\) Q \(\equiv\) (P \(\Rightarrow\) Q) \(\wedge\) (Q \(\Rightarrow\) P)
 
 \(\equiv\) (\(\neg\)P \(\vee\) Q) \(\wedge\) (\(\neg\)Q \(\vee\) P)
 
 \(\neg\)P = ``\(\sqrt{e+3}\) is not a vowel". This statement is true.
 
 So (\(\neg\)P \(\vee\) Q) is true.
 
 Also, \(\neg\)Q = `` \(\pi\) is not an integer". This statement is also true.
 
 So (\(\neg\)Q \(\vee\) P) is true as well.
 
 Therefore, (\(\neg\)P \(\vee\) Q) \(\wedge\) (\(\neg\)Q \(\vee\) P) is true
 
 Thus, P \(\Leftrightarrow\) Q is true.
 
 The statement is true.\\\\
 
 
 (b)\\
 
 Let P = ``2 + 2 = 4" and Q = ``0 $>$ 1".
 
 The statement claims that P \(\Rightarrow\) Q.
 
 P \(\Rightarrow\) Q \(\equiv\) \(\neg\)P \(\vee\) Q
 
 \(\neg\)P = ``2 + 2 \(\neq\) 4", which is false.
 
 Also, since 0 $<$ 1, Q is false.
 
 \(\neg\)P \(\vee\) Q \(\equiv\) \(\mathbb{F}\) \(\vee\) \(\mathbb{F}\) \(\equiv\) \(\mathbb{F}\)
 
 Thus,  P \(\Rightarrow\) Q is false.
 
 The statement is false.\\\\
 
 (c)\\
 
 Let P = ``(a)", Q = ``(b)" and R = ``pigs cannot fly"
 
 The statement claims that (P \(\Rightarrow\) Q) \(\Rightarrow\) R.
 
 P \(\Rightarrow\) Q \(\equiv\) \(\neg\)P \(\vee\) Q,
 
 From (a) and (b), we know that (a) is true and (b) is false. 
 
 Therefore, P is true and Q is false.
 
 \(\neg\)P is false.
 
 P \(\Rightarrow\) Q \(\equiv\) \(\neg\)P \(\vee\) Q \(\equiv\) \(\mathbb{F}\) \(\vee\) \(\mathbb{F}\) \(\equiv\) \(\mathbb{F}\) 
 
 Then,
 
 (P \(\Rightarrow\) Q) \(\Rightarrow\) R
 
 \(\equiv\) \(\mathbb{F}\) \(\Rightarrow\) R
 
 \(\equiv\) \(\neg\) \(\mathbb{F}\) \(\vee\) R
 
 \(\equiv\) \(\mathbb{T}\) \(\vee\) R
 
 \(\equiv\) \(\mathbb{T}\)
 
 The statement is true.
 
\end{solution}

\break

\begin{problem}
Symbolize the following English sentences in logic, using the abbreviation scheme provided.

(a) ``Thunder only happens when it's raining."

T: thunder happens; R: it's raining\\

(b) ``For every positive integer n there is a prime number that is bigger than n but at most 2n."

I(x) : x is a positive integer; P(x) : x is a prime number; B(x,y) : x is bigger than y.\\

(c) ``Goldbach's Conjecture is true if every even integer greater than 2 can be written as the sum of two primes."

G : Goldbach's Conjecture is true; E(x) : x is an even integer; T(x) : x is greater than 2; P(x) : x is the sum of two primes.\\\\
\end{problem}

\begin{solution}
Solution for problem 4:\\

(a)\\

R \(\Rightarrow\) T\\\\

(b)\\

\(\forall\) n I(n) \(\exists\) x [P(x) \(\wedge\) B(x,n) \(\wedge\) \(\neg\) B(x,2n)]\\\\

(c)\\

\(\forall\) x [(E(x) \(\wedge\) T(x) \(\wedge\) P(x)) \(\Rightarrow\) G]




\end{solution}

\end{document}