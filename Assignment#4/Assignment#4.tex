\documentclass{article} 

\usepackage{amsmath,amsthm,amssymb}

\newtheorem{problem}{Problem} 

\theoremstyle{definition} 

\newtheorem*{solution}{Solution} 

\begin{document} \title{Assignment 4} 

\author{Weishi Wang, ID 260540022} 

\date{\today}

\maketitle

\begin{problem} 
\textbf{Division algorithm}

The division algorithm states that for any a,b \(\in\) \(\mathbb{Z}\) (b \(\neq\) 0) there exist q,r \(\in\) \(\mathbb{Z}\) such that a = qb + r and 0 \(\leq\) r \(<\) \(|b|\); furthermore, these q, r are unique for a, b. We proved this when a, b \(>\) 0. Prove that q, r exist for all a, b. Hints: (1) You may use the fact that the statement holds when a, b \(>\) 0 as a tool without proving it and (2) you will need to consider cases.\\\\
\end{problem}





\begin{solution}
We have proven the case a,b\(>\)0. 

Let's consider other cases.\\

\textbf{Case 1:} a\(>\)0 and b\(<\)0:

Look at the following multiple of b:

0, -b, -2b, -3b,...

There is some multiple of b that is greater than a. [ex: -(2a)b = (-2b)a \(\geq\) a]

Let B = \{ kb \(|\) k\(\in\) \(\mathbb{Z}\), kb \(>\) a\}

By the well ordering principle, B. has a smallest element, call it q-1. (q\(<\)0)

Then, since (q-1)b is the smallest element that is greater than a, qb must be smaller or equal to a:

qb \(\leq\) a \(<\) (q-1)b

let r = a - qb

Then,

0 \(\leq\) r \(<\) (q-1)b - 1b

0 \(\leq\) r \(<\) -b

Since b \(<\) 0, we have:

0 \(\leq\) r \(<\) \(|b|\)\\



\textbf{Case 2:} a=0 and b\(>\)0:

if a=0, then r = -qb

let q = 0, then r = 0

Then 0 \(\leq\) r \(<\) b is satisfied.\\


\textbf{Case 3:} a=0 and b\(<\)0:

Same proof as in Case 2.\\


\textbf{Case 4:} a\(<\)0 and b\(>\)0:

Let B = \{kb \(|\) k \(\in\) \(\mathbb{Z}\), kb \(<\) -a\}

The well ordering principle says that there exists a least integer greater than some number.

Therefore, in this set, it must exist a largest integer smaller than -a (which is positive).

Find the greatest element in B and call it (-q-1)b. (q\(>\)0)

Since (-q-1)b is the greatest element smaller than -a, (-q-1)b+b must be greater or equal to -a:

(-q-1)b \(<\) -a \(\leq\) (-q-1+1)b

(-q-1)b \(<\) -a \(\leq\) -qb

Since a = qb +r, then -a = -qb -r.

(-q-1)b \(<\) -qb-r \(\leq\) -qb

(-q-1)b + qb \(<\) -r \(\leq\) -qb + qb

-b \(<\) -r \(\leq\) 0

0 \(\leq\) r \(<\) b

Since b \(>\) 0, b = \(|b|\)

0 \(\leq\) r \(<\) \(|b|\)\\



\textbf{Case 5:} a\(<\)0 and b\(<\)0:

Let B = \{kb \(|\) k \(\in\) \(\mathbb{Z}\), kb \(<\) -a\}

Find the greatest element in B and call it (-q+1)b. (q\(>\)0 and b\(<\)0)

Since (-q+1)b is the greatest element smaller than -a, (-q+1)b-b must be greater or equal to -a (Since b\(<\)0):

(-q+1)b \(<\) -a \(\leq\) -qb

(-q+1)b \(<\) -a \(\leq\) -qb

Since a = qb +r, then -a = -qb -r.

(-q+1)b \(<\) -qb-r \(\leq\) -qb

(-q+1)b + qb \(<\) -r \(\leq\) -qb + qb

b \(<\) -r \(\leq\) 0

0 \(\leq\) r \(<\) -b

Since b \(<\) 0, -b = \(|b|\)

0 \(\leq\) r \(<\) \(|b|\)\\

Moreover, b cannot be 0, therefore all cases are considered.




\end{solution}




\begin{problem}

\textbf{Divisors}

(a) Find gcd(2018, 240), and express you answer as a linear combination of 2018 and 240 (that is, find r, s \(\in\) \(\mathbb{Z}\) such that gcd(2018, 240) = 2018r + 240s).\\

(b) Let k be a positive integer. Show that if a and b are relatively prime integers, then gcd(a+kb,b+ka) divides \(k^2\)-1. Hint: Consider two linear combinations of a + kb and b + ka.\\

(c) Suppose n, m, p \(\in\) \(\mathbb{N}\), p a prime, where p \(|\) n, m \(|\) n, and p \(\nmid\) m. Either prove that p divides \(\frac{n}{m}\) or provide a counterexample to show that it doesn't. Make sure to address whether or not "p divides \(\frac{n}{m}\)" even makes sense.\\

\end{problem}

\begin{solution}
(a) Apply Euclidean Algorithm:

2018 = 8 \(\times\) (240) + 98

240 = 2 \(\times\) 98 + 44

98 = 2 \(\times\) 44 + 10

44 = 4 \(\times \)10 + 4

10 = 2 \(\times\) 4 + 2

4 = 2 \(\times\) 2 + 0

Thus, gcd(2018, 240) = 2\\

2 = 10 - 2 \(\times\) 4

= 10 - 2[44 - 4(10)]

= 9(10) - 2(44)

= 9(98-2(44)) - 2(44)

= 9(98) - 20(44)

= 9(98) - 20(240 - 2(98))

= 49(98) - 20(240)

= 49(2018 - 8(240)) - 20(240)

= 49 \(\times\) 2018 - 412 \(\times\) 240\\

Thus, 

2 = (49 \(\times\) 2018) - (412 \(\times\) 240)\\\\




(b)
Consider the lemma: if g \(|\) a and g \(|\) b, then g \(|\) xa + yb, \(\forall\) x,y \(\in\)\(\mathbb{Z}\)

Proof: g \(|\) a, then pg = a, p \(\in\)\(\mathbb{Z}\)

g \(|\) b, then qg = b, q\(\in\)\(\mathbb{Z}\)

xa + yb = xpg + yqg, x,y,p,q\(\in\)\(\mathbb{Z}\)

xa + yb = (px + qy)g, x,y,p,q\(\in\)\(\mathbb{Z}\)

so g \(|\) (xa+yb).

Which means that g divides any linear combination of a and b.\\

Now consider the question:

Let g = gcd(a+kb, b+ka)

consider the linear combination -(a+kb)+k(b+ak).

By lemma, we know that g \(|\) [-(a+kb)+k(b+ak)].

\(\Rightarrow\) g \(|\) [a(\(k^2-1\))].\\


consider another linear combination k(a+kb)-(b+ak).

By lemma, we know that g \(|\) [k(a+kb)-(b+ak)].

\(\Rightarrow\) g \(|\) [b(\(k^2-1\))].\\

Thus, g \(|\) [a(\(k^2-1\))] and g \(|\) [b(\(k^2-1\))].\\

There are 4 possibilities:

(1) g \(|\) (\(k^2-1\)) and g \(|\) a

(2) g \(|\) (\(k^2-1\)) and g \(|\) b 

(3) g \(|\) (\(k^2-1\))

(4) g \(|\) b and g \(|\) a\\

However, (4) is not possible since a and b are relatively prime, g cannot divide both of them.

Only (1), (2) and (3) are possible.

They all imply that g \(|\) (\(k^2-1\)).

Therefore, gcd(a+kb,b+ka) divides \(k^2\)-1.\\\\


















(c)
We divide by p and m, so p,m\(\neq\)0.

if n=0, then any number divides n. so p \(|\) \(\frac{n}{m}\) \(\Rightarrow\)  p \(|\) 0, which is always true.\\

The problem states that n,p,m \(\in\) \(\mathbb{N}\), which does not include 0. So we don't really need to consider cases, but it does not affect the solution.\\

if n\(\neq\) 0:

p\(\nmid\)m and p is prime means that gcd(p,m) = 1.

So there are no components in p and m can be canceled.

p \(|\) n, m \(|\) n, and gcd(p,m)=1 means that n must be composed of at least one p and one m.

This implies that pm\(|\)n.

k(pm) = n, k\(\in\)\(\mathbb{Z}\)

kp = \(\frac{n}{m}\), k\(\in\)\(\mathbb{Z}\)

\(\Rightarrow\) p \(|\) \(\frac{n}{m}\).\\


In addition, p divides \(\frac{n}{m}\) makes sense when \(\frac{n}{m}\) is an integer. The problem states that m \(|\) n, therefore \(\frac{n}{m}\) must be integer when m \(\neq\) 0.  \\\\



\end{solution}
















\begin{problem}
\textbf{Congruence and modular arithmetic}

(a) Let k \(\in\) \(\mathbb{Z}\) \(\setminus\)\{0\}. Prove that ka \(\equiv\) kb (mod kn) if and only if a \(\equiv\) b (mod n).\\

(b) Prove that if a \(\equiv\) b (mod n), then gcd(a,n) = gcd(b,n).\\

(c) Show that 1806\(^6\)\(^2\)\(^3\)\(^6\) \(\equiv\) 1 (mod 17).\\


\end{problem}



\begin{solution}
(a) ka \(\equiv\) kb (mod kn)

\(\Leftrightarrow\) kn\(|\)(ka -kb)

\(\Leftrightarrow\) (ka - kb) = xkn, x \(\in\) \(\mathbb{Z}\)

\(\Leftrightarrow\) (a-b) = xn, x \(\in\) \(\mathbb{Z}\) (Since k \(\neq\) 0)

\(\Leftrightarrow\) n\(|\)(a-b)

\(\Leftrightarrow\) a \(\equiv\) b (mod n)\\\\


(b) a \(\equiv\) b (mod n)

\(\Rightarrow\) n\(|\)(a-b) \(\Rightarrow\) (a - b) = kn, k \(\in\) \(\mathbb{Z}\). (*)

Let \(g_1\) = gcd(a,n) and \(g_2\) = gcd(b,n)

Divide \(g_1\) on both side of (*):

\(\frac{a-b}{g_1}\) = \(\frac{kn}{g_1}\)

\(\frac{b}{g_1}\) = \(\frac{a}{g_1}\) - \(\frac{kn}{g_1}\)

since \(g_1\) = gcd(a,n)

\(\Rightarrow\) \(g_1\)\(|\)a and \(g_1\)\(|\)n

Thus, \(\frac{a}{g_1}\), \(\frac{kn}{g_1}\) \(\in\) \(\mathbb{Z}\)

So, \(\frac{b}{g_1}\) \(\in\) \(\mathbb{Z}\)

\(\Rightarrow\) \(g_1\)\(|\)b

This means that \(g_1\)\(|\)b and \(g_1\)\(|\)n

but \(g_2\) is the gcd(b,n), so \(g_1\) \(\leq\) \(g_2\)\\

Similarly, divide (*) by \(g_2\), we obtain:

\(\frac{a}{g_2}\) = \(\frac{b}{g_2}\) + \(\frac{kn}{g_2}\)

\(g_2\)\(|\)b and \(g_1\)\(|\)n

So,  \(g_2\)\(|\)a

This means that \(g_2\)\(|\)a and \(g_2\)\(|\)n

but \(g_1\) is the gcd(a,n), so \(g_2\) \(\leq\) \(g_1\)\\

Combine both result, we conclude that \(g_1\) = \(g_2\).\\\\


(c) \(1086^{6236}\) (mod 17)

\(\equiv\) (17\(\times\)106 + 4)\(^{6236}\) (mod 17)

\(\equiv\) 4\(^{6236}\) (mod 17)

\(\equiv\) (4\(^{2}\))\(^{3118}\) (mod 17)

\(\equiv\) 16\(^{3118}\) (mod 17)

\(\equiv\) (17-1)\(^{3118}\) (mod 17)

\(\equiv\) (-1)\(^{3118}\) (mod 17)

Since 3118 is an even number, (-1)\(^{3118}\) = 1.

Therefore \(1086^{6236}\) \(\equiv\) 1 (mod 17).
\end{solution}
  
\end{document}